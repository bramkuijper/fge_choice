\documentclass{article}
\usepackage{amsmath,amssymb}
\usepackage{ragged2e}
\usepackage{IEEEtrantools}
\usepackage{xcolor}
\usepackage[authoryear]{natbib}
\bibliographystyle{plainnat}
\usepackage[a4paper, margin=2.5cm]{geometry}
\usepackage{graphicx}
\usepackage{hyperref}
\usepackage{comment}
\hypersetup{%
  pdftitle={Supplementary Model Information},
  colorlinks=true,%
  linkcolor=blue,%
  filecolor=blue,%
  citecolor=blue,%
  urlcolor=blue,%
}



\renewcommand\theequation{S\arabic{equation}}
\renewcommand\thefigure{S\arabic{figure}}
\setcounter{equation}{0}
\setcounter{figure}{0}
\title{Supplementary Model Information}
\author{}
\date{\vspace{-2.5em}}
\newcommand\SG[1]{\textcolor{orange}{[#1]}}
\newcommand\BK[1]{\textcolor{magenta}{[#1]}}

\usepackage[mathlines]{lineno}

% symbols so that we can more easily decide whether we want to use 
% \badFGE vs G, M vs G, M13s vs M13d, etc etc.
\newcommand{\badFGE}{M}
\newcommand{\goodFGE}{G}
\newcommand{\superinfected}{MG}

% similarly symbols for sensitive and CRISPR immune
\newcommand{\sensitive}{\mathrm{n}}
\newcommand{\immune}{\mathrm{c}}

% to prevent gaps in line numbering when using amsmath
% http://phaseportrait.blogspot.com/2007/08/lineno-and-amsmath-compatibility.html
\newcommand*\patchAmsMathEnvironmentForLineno[1]{%
  \expandafter\let\csname old#1\expandafter\endcsname\csname #1\endcsname
  \expandafter\let\csname oldend#1\expandafter\endcsname\csname end#1\endcsname
  \renewenvironment{#1}%
     {\linenomath\csname old#1\endcsname}%
     {\csname oldend#1\endcsname\endlinenomath}}% 
\newcommand*\patchBothAmsMathEnvironmentsForLineno[1]{%
  \patchAmsMathEnvironmentForLineno{#1}%
  \patchAmsMathEnvironmentForLineno{#1*}}%
\AtBeginDocument{%
\patchBothAmsMathEnvironmentsForLineno{equation}%
\patchBothAmsMathEnvironmentsForLineno{align}%
\patchBothAmsMathEnvironmentsForLineno{flalign}%
\patchBothAmsMathEnvironmentsForLineno{alignat}%
\patchBothAmsMathEnvironmentsForLineno{gather}%
\patchBothAmsMathEnvironmentsForLineno{multline}%
\patchBothAmsMathEnvironmentsForLineno{IEEEeqnarray}%

}





%\linenumbers
\begin{document}
\maketitle

Here we develop an ecological model to assess whether CRISPR-immune (CI) strains
which resist beneficial foreign genetic elements (MGEs) can increase in
frequency, while competing with sensitive strains without such
resistance.

\section{Strains and MGEs}\label{sec:strains-and-fges}

We consider two types of MGE: the first MGE is denoted as $\badFGE$ (for mediocre MGE) and is deleterious or modestly beneficial, as it provides protection against a single
antibiotic. The second MGE is denoted as $\goodFGE$ (for good MGE) and is highly beneficial:
it provides protection against the same antibiotic as $\badFGE$ does, but
also against another antibiotic.

We consider two different strains of hosts: the first strain is a so-called 
sensitive strain (denoted by subscript $\sensitive$) that is equally prone to
be infected by either MGE. The second strain is a CRISPR immune strain 
(denoted by subscript $\immune$) that resists infection by the $\badFGE$ MGE with
probability $\pi$. Here, resistance reflects the probability
that an MGE of type $\badFGE$ is detected and degraded by CRISPR (see
also \citet{Gandon2013}). By contrast, hosts of the immune strain are 
infected by the $\goodFGE$ MGE at rates identical to that of 
the sensitive strain. 

\section{Dynamics\label{sec:dynamics}}
Let $S_{\sensitive}$ and $S_{\immune}$ be the numbers of susceptible
sensitive and CRISPR immune hosts respectively, whereas
$I_{\sensitive i}$ and $I_{\immune i}$ are the numbers of sensitive and 
CRISPR immune hosts infected with an MGE of each type  $i \in (\badFGE,\goodFGE)$. 

We use the following notation for the sums of infected hosts $I_{\badFGE} = I_{\sensitive\badFGE} + I_{\immune\badFGE}$ 
reflects the total number number of hosts infected with the bad MGE,
whereas $I_{\goodFGE} = I_{\sensitive\goodFGE} + I_{\immune\goodFGE}$ reflects the total number of hosts infected with the good MGE. Similarly, $I = I_{\badFGE} + I_{\goodFGE}$ reflects the total number of infected hosts. Finally, the total population size is given by $N=S+ I$. The notation is provided in Table \ref{tab:notation}.

\begin{table}
    \begin{center}
        \begin{tabular}{p{2cm}p{10cm}}
\hline 
Symbol & Definition\tabularnewline
\hline 
    $\immune$ & CRISPR immune host allele (reduced infection by $\badFGE$ MGEs) \tabularnewline
    $\sensitive$ & Sensitive host allele (prone to infection by any MGE) \tabularnewline
    $\badFGE$ & Modestly beneficial MGE \tabularnewline
    $\goodFGE$ & Highly beneficial MGE \tabularnewline
    $S_{i}$ & Density of susceptible hosts carrying allele $i \in \{\immune,\sensitive\}$ \tabularnewline
    $I_{ij}$ & Density of infected hosts carrying resistance allele $i \in \{\immune,\sensitive\}$ and infected by an MGE of type $j \in \{\badFGE,\goodFGE\}$  \tabularnewline 
    $f_{ij}$ & frequency infected hosts carrying resistance allele $i \in \{\immune,\sensitive\}$ and infected by an MGE of type $j \in \{\badFGE,\goodFGE\}$  \tabularnewline 
    $I$ & Total density of all infected hosts, $I = \sum_{ij} I_{ij}$   \tabularnewline 
    $N$ & Total population density \tabularnewline 
            $p_{\immune}$ & Frequency of the CRISPR immune allele in infected hosts, $p_{\immune} = (I_{\immune \badFGE} + I_{\immune \goodFGE})/I$ where $p_{\sensitive} = 1-p_{\immune}$ \tabularnewline
            $q_{\goodFGE}$ & Frequency of the $\goodFGE$ MGE, $q_{\goodFGE} = (I_{\immune \goodFGE} + I_{\sensitive \goodFGE})/I$ where $q_{\badFGE} = 1 - q_{\sensitive}$ \tabularnewline
            $D$ & Linkage disequilibrium measuring the association between host alleles and MGE type (i.e., the excess in frequency of hosts with $\immune$ and $\goodFGE$ and hosts with $\sensitive$ and $\badFGE$ relative to other hosts). \tabularnewline
            $b_{S}, b_{\badFGE}, b_{\goodFGE}$ & Birth rates of susceptible hosts, hosts infected with $\badFGE$, hosts infected with $\goodFGE$ respectively \tabularnewline
            $d_{S}, d_{\badFGE}, d_{\goodFGE}$ & Death rates of susceptible hosts, hosts infected with $\badFGE$, hosts infected with $\goodFGE$ respectively \tabularnewline
            $r_{S}, r_{\badFGE}, r_{\goodFGE}$ & Growth rates of susceptible hosts, hosts infected with $\badFGE$, hosts infected with $\goodFGE$ respectively \tabularnewline 
            $\gamma_{\badFGE}, \gamma_{\goodFGE}$ & Recovery rates of hosts infected with $\badFGE$, $\goodFGE$ respectively \tabularnewline
            $\beta_{\badFGE}, \beta_{\goodFGE}$ & Transmission rates of $\badFGE$, $\goodFGE$ MGEs respectively \tabularnewline
            $k$ & Cost to host of carrying a CRISPR immune allele, measured by a reduction in growth rate $r_{i}$ \tabularnewline
            $\pi$ & Probability of resisting an infection by an $\badFGE$ MGE \tabularnewline
            \hline 
\end{tabular}
\caption{\label{tab:notation}}
    \end{center}
\end{table}


The combined dynamics of CRISPR immune vs sensitive genotypes and both types of MGE is then given by the following system of differential equations. Ignoring coinfections, we have
\begin{IEEEeqnarray}{rl}
    %
% dS_{sensitive}/dt
    \frac{\mathrm{d} S_{\sensitive}}{\mathrm{d}t} &= b_{S} S_{\sensitive} + \gamma_{\badFGE} I_{\sensitive\badFGE} + \gamma_{\goodFGE} I_{\sensitive\goodFGE} - \left [ d_{S} + \beta_{\goodFGE} I_{\goodFGE} + \beta_{\badFGE} I_{\badFGE} \right ] S_{\sensitive} \label{eq:dSsensitivedt} \\
    %
    % dS_{immune}/dt
    \frac{\mathrm{d}S_{\immune}}{\mathrm{d}t} &= (1-k) b_{S} S_{\immune} +  \gamma_{\badFGE} I_{\immune\badFGE} + \gamma_{\goodFGE} I_{\immune\goodFGE} - \left [ (1-k) d_{S} + 
    \left (1-\pi \right ) \beta_{\badFGE} I_{\badFGE} + \beta_{\goodFGE} I_{\goodFGE} \right ] S_{\immune} \label{eq:dSimmunedt} \\
    %
% dIsensitiveBdt
    \frac{\mathrm{d}I_{\sensitive\badFGE}}{\mathrm{d}t} &= b_{\badFGE} I_{\sensitive\badFGE} + \beta_{\badFGE} S_{\sensitive} I_{\badFGE} - \left [ d_{\badFGE} + \gamma_{\badFGE}  \right ] I_{\sensitive\badFGE} 
    \\ 
    %
% dIsensitiveGdt
    \frac{\mathrm{d}I_{\sensitive\goodFGE}}{\mathrm{d}t} &= b_{\goodFGE} I_{\sensitive\goodFGE} + \beta_{\goodFGE} S_{\sensitive} I_{\goodFGE} - \left [ d_{\goodFGE} + \gamma_{\goodFGE} \right ] I_{\sensitive\goodFGE} \label{eq:dIpGdt}
    \\
% dIimmuneBdt
    \frac{\mathrm{d}I_{\immune\badFGE}}{\mathrm{d}t} &= (1-k) b_{\badFGE} I_{\immune\badFGE} + \left ( 1 - \pi\right ) \beta_{\badFGE} S_{\immune} I_{\badFGE} - \left [ (1-k) d_{\badFGE} + \gamma_{\badFGE} \right ] I_{\immune\badFGE}
    \\
% dIimmuneGdt
    \frac{\mathrm{d}I_{\immune \goodFGE}}{\mathrm{d}t} &= (1-k) b_{\goodFGE} I_{\immune \goodFGE} + \beta_{\goodFGE} S_{\immune} I_{\goodFGE} - \left [ (1-k) d_{\goodFGE} + \gamma_{\goodFGE} \right ] I_{\immune\goodFGE} \label{eq:dIcBdt}
\end{IEEEeqnarray} where the first term $b_{x}$ of each equation reflects the fecundity of an host of type $x \in {S,\badFGE,\goodFGE}$ given by the function $b_{x} = F_{x} \left( 1-\kappa N \right)$. 
Here, $F_{x}$ reflects how birth rates differ between susceptible hosts vs hosts infected by $\badFGE$ or $\goodFGE$ MGEs. Moreover, fecundity is assumed to be density
dependent, measured by intensity $\kappa$, where
$N = S_{\sensitive} + S_{\immune} +I_{\sensitive\goodFGE} + I_{\sensitive\badFGE} + I_{\immune\goodFGE} + I_{\immune\badFGE}$ is the total
density of hosts (see eq. [\ref{eq:dNdt}]) below for the dynamics of $N$. 

Next, the term $\gamma_{i} I_{ji}$ in eqns \eqref{eq:dSsensitivedt} and
\eqref{eq:dSimmunedt} reflects the loss of MGEs of type $i$ by infected hosts of type $j$. The terms $[d_{S} + \beta_{\goodFGE}I_{\goodFGE} + \beta_{\badFGE} I_{\badFGE}]S_{\sensitive}$ and
$[(1-k)d_{S} + (1-\pi)\beta_{\badFGE}I_{\badFGE} + \beta_{\goodFGE} I_{\goodFGE}]S_{\immune}$ reflect removal of
sensitive and CRISPR immune hosts respectively, due to (i) mortality at rate
$d_{i}$, (ii) infection by a $\badFGE$ MGE at rate $\beta_{\badFGE}I_{\badFGE}$ and (iii)
infection by a $\goodFGE$ MGE at rate $\beta_{\goodFGE}I_{\goodFGE}$.

Finally, to model costs of CRISPR immunity, we assume that the overall growth rate 
$r_{X} = b_{X} - d_{X}$ of CRISPR immune ($\immune$) hosts is a fraction $1 - k$ of the 
growth rate of sensitive hosts, where $0<k<1$.

The change in $N$ is then given by
\begin{IEEEeqnarray}{rl}
    \frac{\mathrm{d}N}{\mathrm{d}t} &=
    S_{\sensitive} r_{S} + S_{\immune} (1-k) r_{S}
        + I_{\sensitive\goodFGE} r_{\goodFGE}  
        + I_{\sensitive\badFGE} r_{\badFGE}
        + I_{\immune\goodFGE} r_{\goodFGE} (1-k)
        + I_{\immune\badFGE} r_{\badFGE} (1-k) \label{eq:dNdt}
\end{IEEEeqnarray}
with $r_{S} = b_{S} - d_{S}$, $r_{\goodFGE} = b_{\goodFGE} - d_{\goodFGE}$ and $r_{\badFGE} = b_{\badFGE} - d_{\badFGE}$.

\section{Evolutionary dynamics of CRISPR immunity}
To understand the different forces of selection
acting on CRISPR immune hosts, we consider the change in frequency $p_{\immune} = (I_{\immune\badFGE} + I_{\immune\goodFGE})/N$ of infected CRISPR immune hosts, where $p_{\sensitive} = 1-p{\immune}$ reflects the frequency of sensitive infected hosts. Ignoring rare mutations, the change in frequency of CRISPR immunity is then given by
\begin{IEEEeqnarray}{rl}
    \frac{\mathrm{d} p_{\immune}}{\mathrm{d} t} &= 
    \frac{1}{I} \left(
        \frac{\mathrm{d} I_{\immune\goodFGE}}{\mathrm{d}t} + 
        \frac{\mathrm{d} I_{\immune\badFGE}}{\mathrm{d}t} 
        - p_{\immune} \frac{\mathrm{d}I}{\mathrm{d}t} 
    \right) \\
    %
    &= - p_{\immune}\left(1-p_{\immune}\right) k \left [q_{\goodFGE} r_{\goodFGE}  + \left ( 1 - q_{\goodFGE} \right ) r_{\badFGE}  \right ]  \nonumber \\
    %
    & - p_{\immune} S_{\sensitive} \left [ \left ( 1 - q_{\goodFGE} \right ) \beta_{\badFGE} + q_{\goodFGE} \beta_{\goodFGE} \right ] \nonumber \\
    %
    & + \left ( 1 - p_{\immune} \right ) S_{\immune} \left [ \left ( 1 - \pi \right )\left ( 1 - q_{\goodFGE} \right ) \beta_{\badFGE}  + q_{\goodFGE} \beta_{\goodFGE}   \right ] \nonumber \\
    %
    & + D \left [ \gamma_{\badFGE} - \gamma_{\goodFGE} + \left( r_{\goodFGE} - r_{\badFGE} \right) \left ( 1 - \left (1 - p_{\immune} \right) k \right )  \right ]
    \label{eq:dfcdtexpand}
\end{IEEEeqnarray}
The first line of eq. (\ref{eq:dfcdtexpand}) is strictly negative and reflects a decrease in CRISPR immunity due to a growth rate cost $k$ it imposes on its bearer. Note that  $p_{\immune} \left ( 1 - p_{\immune})$ reflects the genetic variance in immunity alleles, followed by a term that reflects natural selection due to costs that accumulate to hosts infected with $\badFGE$ and $\goodFGE$ MGEs respectively.

The second line is again strictly negative and reflects a decrease in frequency of the $\immune$ allele among infected individuals, because susceptible hosts carrying the $\sensitive$ allele become infected. Either a susceptible $\sensitive$ host becomes infected by a $\badFGE$ MGE (at rate $S_{\sensitive}(1-q_{\goodFGE}) \beta_{\badFGE}$) or by a $\goodFGE$ MGE (at rate $S_{\sensitive}q_{\goodFGE}\beta_{\goodFGE}$). 

The third line reflects an increase in frequency of the $\immune$ allele among infected individuals, as susceptible hosts carrying the $\immune$ allele become infected. Such hosts either become infected by a $\badFGE$ MGE (at rate $S_{\immune}(1-\pi)\beta_{\badFGE}\left(1-q_{\goodFGE}\right))$ or by a good $\goodFGE$ MGE (at rate $S_{\immune}\beta_{\goodFGE} q_{\goodFGE}$). 

Finally, the fourth line reflects changes in the frequency of $\immune$ due to an assocation $D$ between  $\immune$ alleles and $\goodFGE$ MGEs (and similarly, an association between $\sensitive$ and $\badFGE$). We find that this association causes $p_{\immune}$ to increase whenever $(r_{\goodFGE} > r_{\badFGE})(1-(1-p_{\immune})k)$, i.e., whenever growth due to infection by the $\goodFGE$ FGE is larger than that of the $\badFGE$ FGE (weighed by a term that discounts any fecundity costs $k$ on carrying $\immune$ rather than $\sensitive$). In addition, $p_{\immune}$ also increases when $\gamma_{\badFGE} > \gamma_{\goodFGE}$ (i.e., when recovery rates are larger for $\badFGE$ MGEs than for $\goodFGE$ MGEs), so that $\immune$ alleles are more likely to profit from $\goodFGE$ than $\badFGE$ MGEs. 

\subsection{Importance of the assocation $D$ between CRISPR immunity and the beneficial MGE}
To illustrate the role of the assocation $D$, let us first consider a scenario where $D = 0$. Comparing terms of line 2 and 3 of equation (\ref{eq:dfcdtexpand}), we obviously have for $\pi>0$ that $(1-\pi)(1-q_{\goodFGE}) \beta_{M} < (1-q_{\goodFGE}) \beta_{M}$. Hence, whenever densities of $\immune$ and $\sensitive$ alleles are initially similar (i.e., $p_{\immune} S_{\sensitive} \approx (1-p_{\immune}) S_{\immune}$), we have
\begin{IEEEeqnarray}{rl}
    \frac{\mathrm{d} p_{\immune}}{\mathrm{d} t} &=  - p_{\immune}\left(1-p_{\immune}\right) k \left [q_{\goodFGE} r_{\goodFGE}  + \left ( 1 - q_{\goodFGE} \right ) r_{\badFGE}  \right ] - p_{\immune} S_{\sensitive} \pi \left ( 1 - q_{\goodFGE} \right ) \beta_{\badFGE} 
\end{IEEEeqnarray}
which is strictly negative. Consequently, the build-up of a positive assocation $D$ between the CRISPR allele $\immune$ and the $\goodFGE$ MGE is essential to achieve $\left . \mathrm{d} p_{\immune} \right / \mathrm{d} t > 0$.
%
\paragraph{Dynamics of $D$}Here we ask under which conditions CRISPR-immunity alleles are more likely than sensitive alleles to become associated with the $\goodFGE$ MGE. The linkage disequilibrium is given by $D = f_{\immune \goodFGE} f_{\sensitive \badFGE} - f_{\immune \badFGE} f_{\sensitive \goodFGE}$, where $f_{ij} = I_{ij} / I \pm D$. Starting from a scenario where there is no such association ($D=0$) we thus investigate when $\left \mathrm{d} D \right / \mathrm{d} t> 0$. A continuous-time equation for the change in $D$ is then given by expanding 
\begin{IEEEeqnarray}{rl}
\frac{\mathrm{d} D}{\mathrm{d} t} &= 
    \frac{\mathrm{d}}{\mathrm{d}t} \left (\frac{I_{\immune\goodFGE}}{I} \right ) f_{\sensitive \badFGE} %
    + \frac{\mathrm{d}}{\mathrm{d}t} \left (\frac{I_{\sensitive \badFGE}}{I} \right ) f_{\immune\goodFGE} %
    - \frac{\mathrm{d}}{\mathrm{d}t} \left (\frac{I_{\immune \badFGE}}{I} \right ) f_{\sensitive \goodFGE} %
    - \frac{\mathrm{d}}{\mathrm{d}t} \left (\frac{I_{\sensitive \goodFGE}}{I} \right ) f_{\immune \badFGE},
\end{IEEEeqnarray}
but the resulting expression is long and uninformative. However, when assuming that we initially start at $D = 0$, the frequencies of infected individuals $f_{ij}$ carrying resistance allele $i$ and infected by an MGE of type $j$ become $f_{ij} = p_{i} q_{j}$. The resulting expression becomes
\begin{IEEEeqnarray}{rl}
    \left . \frac{\mathrm{d} D}{\mathrm{d} t} \right |_{D = 0} &= \left ( 1 - q_{\goodFGE} \right ) q_{\goodFGE} \left [ - \left ( 1 - p_{\immune} \right ) p_{\immune} k \left ( r_{\goodFGE} - r_{\badFGE} \right ) \right . \nonumber \\
    %
        & + p_{\immune} S_{\sensitive} \left( \beta_{\badFGE} - \beta_{\goodFGE} \right ) \nonumber \\
    %
        & \left . + \left ( 1 - p_{\immune} \right ) S_{\immune} \left( \beta_{\goodFGE} - \left ( 1 - \pi \right ) \beta_{\badFGE} \right ) \right ]
    \right ],
\end{IEEEeqnarray}
where the first line reflects a decrease in the association between $\goodFGE$ and $\immune$ because of the costs $k$ of bearing the CRISPR immune allele. The second line reflects an increase in the association because sensitive $\sensitive$ hosts become infected by $\badFGE$ MGEs, rather than by $\goodFGE$ MGEs. Finally, the third line reflects an increase in the association due to infections of $\immune$ hosts by $\goodFGE$ MGEs, relative to $\badFGE$ MGEs. 
\bibliography{refs}
\end{document}
